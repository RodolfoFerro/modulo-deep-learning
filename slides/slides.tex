\documentclass[10pt]{beamer}

\usepackage{pgfpages}
\usepackage[utf8]{inputenc}
%\setbeameroption{show notes on second screen }
\usetheme[
% nojauge,
% nomail,
% rule,
delaunay,
amurmapleblack
]{Amurmaple}

\usepackage[T1]{fontenc}
\usepackage[utf8]{inputenc}
\usepackage{lipsum}

\definecolor{newremark}{rgb}{0.7,0.2,0.2}
\colorlet{AmurmapleRemarkColor}{newremark}

\title[Deep Learning]{Aprendizaje profundo}
\author[R.~Ferro (@rodo\_ferro)]{Rodolfo Ferro}
\subtitle{Módulo 5}
\institute[ENES Unidad León]{Diplomado en Ciencia de Datos\\
	Escuela Nacional de Estudios Superiores, Unidad León}
\date{Julio-agosto, 2024}
\titlegraphic{\includegraphics[width=3cm]{images/logo.png}}
\mail{ferro@cimat.mx}
\webpage{https://rodolfoferro.xyz}
% \collaboration{in collaboration with \LaTeX{}}
\logo{\includegraphics[width=1.8cm]{images/logo.png}}

\begin{document}
    
	\maketitle
	
	\sepframe[title={Tabla de contenidos}]
	
    \frame{\tableofcontents}
	
    %%%%%%%%%%%%%%%%%%%%%%%%%%%%%%%%%%%%%%%%%%%%%    
    % ----------- Intro al DL -------------------
    %%%%%%%%%%%%%%%%%%%%%%%%%%%%%%%%%%%%%%%%%%%%%
	\section{Intro al aprendizaje profundo}
    
    % ----------- Contexto histórico ------------
    \subsection{Contexto histórico}
    
    \begin{frame}{Contexto histórico}{Intro al aprendizaje profundo}
        \begin{block}{Block title}
            Cras viverra metus rhoncus sem.
        \end{block}
    \end{frame}
    
    % ----------- Contexto histórico ------------
    \subsection{Perceptrón multicapa}
    
    \begin{frame}{Contexto histórico}{Intro al aprendizaje profundo}
        \begin{block}{Block title}
            Cras viverra metus rhoncus sem.
        \end{block}
    \end{frame}
    
	\section{Modelado profundo de secuencias}
	\section{Visión computacional profunda}
	\section{Modelado generativo profundo}
	\section{Panorama actual y futuro}
    
    \begin{frame}{Contexto histórico}{Intro al aprendizaje profundo}
        \begin{block}{Block title}
            Cras viverra metus rhoncus sem.
        \end{block}
    \end{frame}
	
\end{document}

%%% Local Variables:
%%% mode: lualatex
%%% TeX-master: Rodolfo Ferro
%%% End:
